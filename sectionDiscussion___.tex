\section{Discussion}


When sample size is not the limiting factor, it is clear that using well-established protocols that involve minimal sequence-specific manipulation of the sample yields the best results, both in terms of reproducibility and linearity of response. However, if it is not practical to collect such relatively large samples, we believe that any of the ``single-cell'' protocols we have tested should perform similarly, and can be used as a drop-in replacement. While preamplification steps do introduce some detectable variance, it is not vastly detrimental to the data quality, and does not introduce obvious sequence-specific biases.

Such methods should be strongly preferred if it is feasible to collect a suitably homogenous sample. While bulk tissues may be a mixture of multiple distinct cell types, this may or may not affect the particular research question an RNAseq experiment is designed to answer.  In our hands, the lower limit of reliable library construction using the Illumina TruSeq kit is approximately 70ng of total RNA; with non precious samples, the practical limit is likely to be even lower. Although we believe there is significant user-to-user variation, it seems unreasonable to expect order-of-magnitude improvements are possible in techniques for precious samples. We suggest that this limit may be related to cDNA binding to tubes or purification beads, but since the quantities are lower than the detection threshold of many standard quality control approaches, we cannot directly verify this, nor do we believe that knowing the precise cause is likely to suggest remediation techniques.


Compared to the regimes these protocols were designed for, we used a relatively large amount of input RNA---1 ng of total RNA---corresponding to approximately 50 nuclei of a mid-blastula transition {\em Drosophila} embryo. Previous studies have shown that this amount of RNA is well above the level where stochastic variation in the number of mRNAs per cell will strongly affect the measured expression of a vast majority of genes \cite{Marinov:2013fm}. It is nevertheless a small enough quantity to be experimentally relevant.  For instance, we have previously dissected single embryos into approximately 12 sections, yielding approximately 10ng per section\cite{Combs:2013jy}, and one could conceivably perform similar experiments on imaginal discs or antennal structures, which contain a similar amount of cells \cite{Klebes:2002ua,Hansson:2000cx}.

One of the more striking results is that costs can be significantly reduced by simply performing smaller reactions, without noticeably degrading data quality.  We do not suspect this will be true for arbitrarily small samples, such as from single cells.  Instead, it is likely only true for samples near the high end of the effective range of the protocol. We have not explored where this result breaks down, and strongly caution others to verify this independently using small pilot experiments before scaling up.
