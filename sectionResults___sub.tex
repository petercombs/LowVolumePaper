\section{Results}


\subsection{Evaluation of existing protocols}


In our hands, the Illumina TruSeq protocol has performed extremely reliably with samples on the scale of ~100ng of total RNA.  However, attempts to create libraries from radically smaller samples yielded extremely low complexity libraries, corresponding to approximately 30-fold PCR duplication of fragments. Anecdotally, less than 5\% of libraries made with at least 90ng of total RNA yielded abnormally low concentrations, which we observed correlated with low complexity. To determine the lower limit of input needed to reliably produce libraries, we attempted to make libraries from 40, 50, 60, 70, and 80 ng of {\em Drosophila} total RNA, each in triplicate.


\begin{table}
\caption{Total cDNA yields from TruSeq libraries made with a given amount of input total RNA.  Yields measured by Nanodrop of cDNA libraries resuspended in 25$\mu L$ of EB. The starred samples were unusually low, and when analyzed with a Bioanalyzer, showed abnormal size distribution of cDNA fragments.}
\begin{center}
\begin{tabular}{|c|c|c|c|}\hline
Amount Input RNA & Replicate A & Replicate B & Replicate C\\\hline
40 ng &  57 ng*  & 425 ng & 672 ng\\
50 ng & 435 ng & 768 ng & 755 ng\\
60 ng & 115 ng* & 663 ng & 668 ng\\
70 ng & 300 ng & 593 ng & 653 ng\\
80 ng & 468 ng & 550 ng & 840 ng\\\hline
\end{tabular}
\end{center}
\label{default}
\end{table}


\subsection{Competitive Comparison of Low-volume RNA protocols}


We first sought to determine whether the low-volume RNAseq protocols available faithfully recapitulate linear changes in abundance of known inputs. 


\subsection{Further modifications to the SMART-seq2 protocol}
