\section{Results}


\subsection{Experiment 1: Evaluation of Illumina TruSeq}

In our hands, the Illumina TruSeq protocol has performed extremely reliably with samples on the scale of ~100ng of total RNA.  However, attempts to create libraries from smaller samples yielded low complexity libraries, corresponding to approximately 30-fold PCR duplication of fragments. Anecdotally, less than 5\% of libraries made with at least 90ng of total RNA yielded abnormally low concentrations, which we observed correlated with low complexity (Data not shown). To determine the lower limit of input needed to reliably produce libraries, we attempted to make libraries from 40, 50, 60, 70, and 80 ng of {\em Drosophila} total RNA, each in triplicate.

\begin{table}[htdp]
\caption{Total cDNA yields from TruSeq libraries made with a given amount of input total RNA.  Yields measured by Nanodrop of cDNA libraries resuspended in 25$\mu L$ of EB. The italicized samples were unusually low, and when analyzed with a Bioanalyzer, showed abnormal size distribution of cDNA fragments.}
\begin{center}
\begin{tabular}{|c|c|c|c|}\hline
Amount Input RNA & Replicate A & Replicate B & Replicate C\\\hline
40 ng & {\em  57 ng}  & 425 ng & 672 ng\\
50 ng & 435 ng & 768 ng & 755 ng\\
60 ng & {\em 115 ng} & 663 ng & 668 ng\\
70 ng & 300 ng & 593 ng & 653 ng\\
80 ng & 468 ng & 550 ng & 840 ng\\\hline
\end{tabular}
\end{center}
\label{table:truseqtitration}
\end{table}



\subsection{Experiment 2: Competitive Comparison of Low-volume RNA protocols}

We first sought to determine whether the low-volume RNAseq protocols available faithfully recapitulate linear changes in abundance of known inputs. We generated synthetic spike-ins by combining {\em D. melanogaster} and {\em D. virilis} total RNA in known, predefined proportions of 0, 5, 10, and 20\% {\em D. virilis} RNA. For each of the low-volume protocols, we used 1ng of total RNA as input, whereas for the TruSeq protocol we used 100ng.

Although pre-defined mixes of spike-in controls are commercially available, we felt it was important to ensure that a given protocol would function reproducibly with natural RNA, which almost certainly has a different distribution of 6-mers. Furthermore, our spike-in sample more densely covers the approximately $10^5$ fold coverage typical of RNA abundances.  It should be noted, however, that our sample is not directly comparable to any other standards, nor is the material of known strandedness.  We assumed that the majority of each sample is from the standard annotated transcripts, but did not verify this prior to library construction and sequencing.

The different protocols had a variation in yield of libraries from between 6 fmole (approximately 3.6 trillion molecules) and 2,400 femtomoles, with the TruSeq a clear outlier at the high end of the range, and the other protocols all below 200 fmole (Table \ref{tab:round1protocols}).  All of these quantities are sufficient to generate hundreds of millions of reads---far more than is typically required for an RNA-seq experiment. We pooled the samples, attempting equimolar fractions in the final pool; however, due to a pooling error, we generated significantly more reads than intended for the TruSeq protocol, and correspondingly fewer in the other protocols. Unless otherwise noted, we therefore sub-sampled the mapped reads to the lowest number of mapped reads in any sample in order to provide a fair comparison between protocols. 

We were primarily interested in the fold-change of each gene across the four samples, rather than the absolute abundance of any particular gene. Therefore, after mapping and gene quantification, we normalized the abundance $A$ of every gene $i$ across the $j=4$ samples by a weighted average of the quantity $Q_j$ of {\em D. virilis} in sample $j$, as show in equation \ref{eqn:norm}.  Thus, within a given gene, a linear fit of $A_ij$ vs $Q_j$ should have a slope of one and an intercept of zero.

\begin{equation} \label{eqn:norm}
\hat{A}_{ij} = A_{ij} \div \frac{\sum_j Q_j A_{ij}}{\sum_j (Q_j)^2}  
\end{equation}


\subsection{Experiment 3: Further modifications to the SMART-seq2 protocol}

Although the SMART-seq2 was the cheapest of the protocols, we wondered whether it could be performed even more cheaply without compromising data quality.  This would enable us to include more biological replicates in subsequent experiments.  In the original protocol, we noticed that roughly 60\% of the cost came from the Nextera XT reagents.  Thus, this seemed to be the most productive point to target.  

