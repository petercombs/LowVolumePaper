\subsection{Experiment 3: Further modifications to the SMART-seq2 protocol}

Although the SMART-seq2 was the cheapest of the protocols, we wondered whether it could be performed even more cheaply without compromising data quality.  This would enable us to include more biological replicates in the future experiments for which we are evaluating these protocols.  In the original protocol, we noticed that roughly 60\% of the cost came from the Nextera XT reagents.  Thus, reducing the cost of tagmentation was the obvious goal to target.  

We made additional libraries, again starting with 1ng of total RNA.  We amplified a single set of spike-in samples with 0, 5, 10, and 20\% {\em D. virilis} total RNA as in experiment 2, and made a single an additional sample with 1\% {\em D. virilis} RNA. Starting at the point in the SMART-seq2 protocol where tagmentation was started, we performed reactions in volumes $2.5\times$ and $5\times$ smaller, using proportionally less cDNA as well.  Due to the low total yield, we increased the number of enrichment cycles from 6 to 8 (see methods). 

When normalized to the same number of reads as in experiment 2, the protocols with diluted Nextera reagents performed effectively identically: for instance, the mean correlation coefficients were in both cases $0.96 \pm 0.05$ (Fig. \ref{fig:dilutions} and Table \ref{tab:fits}).  This is despite the additional cycles of enrichment, which improved yield.

Because we used a common set of pre-amplified cDNA samples that was performed in a distinct pre-amplification from experiment 2, we can estimate the contribution of that pre-amplification to the overall variation. If, in fact, the pre-amplification is a major contributor to the variation, then we would expect to find that the correlation between, for instance, the slopes of two runs of the same experiment with different pre-amplifications would be significantly lower than the correlation between the slopes of two runs using the same pre-amplified cDNA pools. 

Unsurprisingly, the sets of samples that used the same preamplification were more correlated with each other than with the set of samples that used a separate pre-amplification (Fig. \ref{fig:noisesource}).  By analogy to dual-reporter expression studies\cite{Elowitz:2002hb}, we term variation along the diagonal ``extrinsic noise'' ($\eta_{ext} = \mbox{std}(m_1 + m_2)$), and variation perpendicular to the diagonal ``intrinsic noise'' ($\eta_{int} = \mbox{std}(m_1 - m_2)$), being intrinsic to the pre-amplification step.  Using that metric, the intrinsic noise is lower for the samples with the same pre-amplification ($\eta_{int} =0.09$) than for the samples with different pre-amplifications ($\eta_{int} =0.16$). Somewhat surprisingly, the extrinsic noise is higher for the samples with the same pre-amplification ($\eta_{ext} = 0.20$ vs $\eta_{ext} = 0.16$), perhaps due to the 2 additional cycles of PCR enrichment.