\section{Methods}


\subsection{RNA Extraction, Library Preparation, and Sequencing}


We performed RNA extraction in TRIzol (Life Technologies, Grand Island, NY) according to manufacturer instructions, except with a higher concentration of glycogen as carrier (20 ng) and a higher relative volume of TRIzol to the expected material (1 mL, as in \cite{Lott:2011cc} and \cite{Combs:2013jy}). We quantified RNA concentrations using a fluorometric Qubit RNA HS assay (Life Technologies). 

TruSeq libraries were prepared with the ``TruSeq RNA Sample Preparation Kit v2'' (Illumina Cat.#RS-122-2001) according to manufacturer instructions, except for the following modifications. All reactions were performed in half the volume of reagents. We find that this increases the effective concentration of RNA and cDNA.  We performed all reactions and cleanups in 8-tube PCR strip tubes, which allowed us to reduce the volume of Resuspension Buffer to minimize volume left behind after each cleanup.

Clontech libraries were prepared with the ``Low Input Library Prep Kit'' (Clontech Cat.#634947). We generated cDNA by using TruSeq reagents until the cDNA synthesis step. Then, we used the Low Input Library Prep Kit to modify the cDNA into sequencing-competent libraries. We believe that a similar cDNA synthesis could be performed using oligo dT Dynabeads, RNA fragmentation reagents, and Superscript II (Life Technologies), for an approximate cost per sample of \$15. 

TotalScript libraries were prepared with the ``TotalScript RNA-Seq Kit'' and ``TotalScript Index Kit'' (Epicentre Cat.#TSRNA1296 and TSIDX12910). We followed the manufacturer's instructions, and used the oligo dT priming option.  We performed the mixed priming option in parallel, which yielded approximately 4-fold more library, but did not sequence them due to concerns of ribosomal contamination.

SMARTseq2 libraries were prepared according to the protocol in Picelli {\em et al.}(2014) \cite{Picelli:2014kg}.  Because we had already extracted and mixed the RNA, we began at step 5 with 3.7 $\mu L$ of dNTPs and 1 $\mu L$ of 37 $\mu M$ oligo dT primer, yielding the same concentration of primer and oligo as originally reported. We used 18 cycles for the preamplification PCR in step 14, added 1ng of cDNA to the Nextera XT reactions in step 28, and used 6 and 8 cycles for the final enrichment in step 33 (experiments 2 and 3, respectively).

Libraries were quantified using a combination of Qubit High Sensitivity DNA (Life Technologies) and Bioanalyzer (Agilent Technologies, Sunnyvale, CA) readings, then pooled to equalize index concentration. Due to a pooling error in experiment 2, the TruSeq libraries were included at much higher abundance. Pooled libraries were then submitted to the Vincent Coates Genome Sequencing Laboratory for 50bp single-end sequencing according to standard protocols for the Illumina HiSeq 2500. Bases were called using HiSeq Control Software v1.8 and Real Time Analysis v2.8.


\subsection{Mapping and Quantification}


Reads were mapped using STAR \cite{Dobin:2012fg} to a combination of the FlyBase reference genome version 5.54 for {\em D. melanogaster} and {\em D. virilis} \cite{McQuilton:2011iq}. We randomly sampled the mapped reads to use an equal number in each sample compared. We used HTSeq (command line options ``\verb|htseq-count --idattr='gene\_name' --stranded=no --sorted=pos|'') to count absolute read abundance per gene \cite{Anders_2014}.
